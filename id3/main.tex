\documentclass[a4paper]{article}
\usepackage{hyperref}
\usepackage{graphicx}
\usepackage{booktabs}
\usepackage{float}
\usepackage{amsmath}

\usepackage{booktabs, multirow} % for borders and merged ranges
\usepackage{soul}% for underlines
\usepackage[table]{xcolor} % for cell colors
\usepackage{changepage,threeparttable} % for wide tables

\begin{document}

\section{ID3 Algorithm}
ID3 algorithm is used to create a decision tree. A tree is created from a data using the attributes. Information is presented in tree. Using the tree you can easily classify,a new data point.

For example given car data as in table:~\ref{table:car_data}, and you want to predict if a car is fast or not given the attributes, Engine, SC/Turbo, Weight, and Fuel Eco.

The car's model does not matter, because it is unique.
%Please add the following packages if necessary:
%\usepackage{booktabs, multirow} % for borders and merged ranges
%\usepackage{soul}% for underlines
%\usepackage[table]{xcolor} % for cell colors
%\usepackage{changepage,threeparttable} % for wide tables
%If the table is too wide, replace \begin{table}[!htp]...\end{table} with
%\begin{adjustwidth}{-2.5 cm}{-2.5 cm}\centering\begin{threeparttable}[!htb]...\end{threeparttable}\end{adjustwidth}
\begin{table}[!htp]\centering
    \caption{Car data}\label{table:car_data}
    \scriptsize
    \begin{tabular}{lrrrrrr}\toprule
        \textbf{Model} & \textbf{Engine} & \textbf{SC/Turbo} & \textbf{Weight} & \textbf{Fuel Eco} & \textbf{Fast} \\\midrule
        Prius          & small           & no                & average         & good              & \textbf{no}   \\
        Civic          & small           & no                & light           & average           & \textbf{no}   \\
        WRX STI        & small           & yes               & average         & bad               & \textbf{yes}  \\
        M3             & medium          & no                & heavy           & bad               & \textbf{yes}  \\
        RS4            & large           & no                & average         & bad               & \textbf{no}   \\
        GTI            & medium          & no                & light           & bad               & \textbf{no}   \\
        XJR            & large           & yes               & heavy           & bad               & \textbf{no}   \\
        S500           & large           & no                & heavy           & bad               & \textbf{no}   \\
        911            & medium          & yes               & light           & good              & \textbf{no}   \\
        Convette       & large           & no                & average         & bad               & \textbf{yes}  \\
        Insight        & small           & no                & light           & good              & \textbf{no}   \\
        RSX            & small           & no                & average         & average           & \textbf{no}   \\
        IS350          & medium          & no                & heavy           & bad               & \textbf{no}   \\
        MR2            & small           & yes               & average         & average           & \textbf{no}   \\
        E320           & medium          & no                & heavy           & bad               & \textbf{no}   \\
        \bottomrule
    \end{tabular}
\end{table}

\subsection{Calculate the Information Gain}

Establish the target classification, is the car fast?
6/15 yes, 9/15 no

Calculate the classification entropy
\begin{equation}
    I_E = -(6/15)\log(6/15) - (9/15)\log(9/15)
\end{equation}
The information gain of the values is:

%%%%%%%%%%%%%%%%%%%%%%%%%%%%%%%%%%%%%%%%%%%%%%%%%%%%%%%%%%%%%%%
% information gain for the first node
%%%%%%%%%%%%%%%%%%%%%%%%%%%%%%%%%%%%%%%%%%%%%%%%%%%%%%%%%%%%%%%%%%%%%

%Please add the following packages if necessary:
%\usepackage{booktabs, multirow} % for borders and merged ranges
%\usepackage{soul}% for underlines
%\usepackage[table]{xcolor} % for cell colors
%\usepackage{changepage,threeparttable} % for wide tables
%If the table is too wide, replace \begin{table}[!htp]...\end{table} with
%\begin{adjustwidth}{-2.5 cm}{-2.5 cm}\centering\begin{threeparttable}[!htb]...\end{threeparttable}\end{adjustwidth}
\begin{table}[!htp]\centering
    \caption{Information Gain of all the features}\label{tab:firstnode }
    \scriptsize
    \begin{tabular}{lrr}\toprule
        \textbf{Feature} & \textbf{Information gain} \\\midrule
        Engine           & 0.004936                  \\
        SC/Turbo         & 0.003959                  \\
        Weight           & 0.113967                  \\
        Fuel Eco         & 0.170951                  \\
        \bottomrule
    \end{tabular}
\end{table}

The feature Fuel Eco has the highest information gain we make it the root node. Under Fuel Eco we can branch on bad, everage, good

We create a filter of the data considering branching on the bad feature of Fuel Eco.
%%%%%%%%%%%%%%%%%%%%%%%%%%%%%%%%%%%%%%%%%%%%%%
% node 2 bad
%%%%%%%%%%%%%%%%%%%%%%%%%%%%%%%%%%%%%%%%%%%%%%%%%

%Please add the following packages if necessary:
%\usepackage{booktabs, multirow} % for borders and merged ranges
%\usepackage{soul}% for underlines
%\usepackage[table]{xcolor} % for cell colors
%\usepackage{changepage,threeparttable} % for wide tables
%If the table is too wide, replace \begin{table}[!htp]...\end{table} with
%\begin{adjustwidth}{-2.5 cm}{-2.5 cm}\centering\begin{threeparttable}[!htb]...\end{threeparttable}\end{adjustwidth}
\begin{table}[!htp]\centering
    \caption{Information gain after the first node, Fuel Eco, branching on bad}\label{firstNode:bad_feature}
    \scriptsize
    \begin{tabular}{lrrrr}\toprule
        \textbf{Engine} & \textbf{SC/Turbo} & \textbf{Weight} & \textbf{Fast} \\\midrule
        small           & yes               & average         & \textbf{yes}  \\
        medium          & no                & heavy           & \textbf{yes}  \\
        large           & no                & average         & \textbf{no}   \\
        medium          & no                & light           & \textbf{no}   \\
        large           & yes               & heavy           & \textbf{no}   \\
        large           & no                & heavy           & \textbf{no}   \\
        large           & no                & average         & \textbf{yes}  \\
        medium          & no                & heavy           & \textbf{no}   \\
        medium          & no                & heavy           & \textbf{no}   \\
        \bottomrule
    \end{tabular}
\end{table}

Then calculate the information gain, and create a node with the highest information gain.
%%%%%%%%%%%%%%%%%%%%%%%%%%%%%%%%%%%%%%%%%%%%%%%%%%%%%%%%%%%%%%%%%%%%%%%%%%%%%%%%%%%
% Information gain on the 2nd node when fuel Eco is bad
%%%%%%%%%%%%%%%%%%%%%%%%%%%%%%%%%%%%%%%%%%%%%%%%%%%%%%%%%%%%%%%%%%%%%%%%%%%%%%%%%%%%%%

%Please add the following packages if necessary:
%\usepackage{booktabs, multirow} % for borders and merged ranges
%\usepackage{soul}% for underlines
%\usepackage[table]{xcolor} % for cell colors
%\usepackage{changepage,threeparttable} % for wide tables
%If the table is too wide, replace \begin{table}[!htp]...\end{table} with
%\begin{adjustwidth}{-2.5 cm}{-2.5 cm}\centering\begin{threeparttable}[!htb]...\end{threeparttable}\end{adjustwidth}
\begin{table}[!htp]\centering
    \caption{Information Gain when Fuel Eco is Bad}\label{ig:fuel-eco_bad}
    \scriptsize
    \begin{tabular}{lrr}\toprule
        \textbf{feature} & \textbf{ig} \\\midrule
        Engine           & 0.19716     \\
        SC/Turbo         & 0.024758    \\
        Weight           & 0.211126    \\
        \bottomrule
    \end{tabular}
\end{table}

From table:~\ref{ig:fuel-eco_bad} Wight has the highest information gain. We create a node for Weight.
We then create a view of the data considering only the remaining features,
%Please add the following packages if necessary:
%\usepackage{booktabs, multirow} % for borders and merged ranges
%\usepackage{soul}% for underlines
%\usepackage[table]{xcolor} % for cell colors
%\usepackage{changepage,threeparttable} % for wide tables
%If the table is too wide, replace \begin{table}[!htp]...\end{table} with
%\begin{adjustwidth}{-2.5 cm}{-2.5 cm}\centering\begin{threeparttable}[!htb]...\end{threeparttable}\end{adjustwidth}
\begin{table}[!htp]\centering
    \caption{Data When weight is average}\label{table:weight_average}
    \scriptsize
    \begin{tabular}{lrrr}\toprule
        \textbf{Engine} & \textbf{SC/Turbo} & \textbf{Fast} \\\midrule
        small           & yes               & \textbf{yes}  \\
        large           & no                & \textbf{no}   \\
        large           & no                & \textbf{yes}  \\
        \bottomrule
    \end{tabular}
\end{table}

Then calculate the information gain %Please add the following packages if necessary:
%\usepackage{booktabs, multirow} % for borders and merged ranges
%\usepackage{soul}% for underlines
%\usepackage[table]{xcolor} % for cell colors
%\usepackage{changepage,threeparttable} % for wide tables
%If the table is too wide, replace \begin{table}[!htp]...\end{table} with
%\begin{adjustwidth}{-2.5 cm}{-2.5 cm}\centering\begin{threeparttable}[!htb]...\end{threeparttable}\end{adjustwidth}
\begin{table}[!htp]\centering
    \caption{Generated by Spread-LaTeX}\label{tab:weight_average }
    \scriptsize
    \begin{tabular}{lrr}\toprule
        \textbf{feature} & \textbf{ig} \\\midrule
        Engine           & 0.251629    \\
        SC/Turbo         & 0.251629    \\
        \bottomrule
    \end{tabular}
\end{table}

The values have a tie,

\end{document}